\chapter{Research}

    \section{User personas}
         The user personas represent Skyline communications customers. more specifically those who use the DataMiner Dashboards application. The user personas can be found in \autoref{user_personas}.
    \section{User stories}
        \begin{itemize}
            \item {}
            \item {}
            \item {}
            \item {}
        \end{itemize}
        \begin{itemize}
            dit moeten er 15 a 20 zijn.
            \item {As a user I can sort columns, so that I immediately see the highs and lows.}
            \item {As a user I can filter the columns, so that irrelevant data is not shown. }
            \item {As a user I can swap columns, so that the most relevant data is immediately visible.}
            \item {As a user I can resize columns, so that they take up less space.}
        \end{itemize}
    \section{Use cases}
        
    \section{DataMiner Dashboards competitors}
        As a company you don't always have to reinvent the wheel. It can be very useful to see how competitors solve certain problems. But who are Skyline DataMiner Dashboards' biggest competitors? How do they \textbf{visualize} large tabular data?
        
        \subsection{Tableau}
        \subsubsection{What is Tableau?}
        Tableau is \underline{the} leading Business Intelligence software (\gls{bis}). With Tableau users can easily retrieve, analyze, transform and report data. It enables users to effortlessly visualize data \textbf{regardless of its format}. The software is very easy to use and does not require any knowledge of coding. What makes Tableau so unique compared to others is their quality of data visualizations and self-service analytics. It was found in 2003 by Christian Chabot, Pat Hanrahan and Chris Stolte. In August 2019, Tableau was acquired by \href{https://www.salesforce.com/}{Salesforce} for a stunning \textbf{15 Billion dollars}.
       \subsubsection{Table} 
         
       
        \subsection{Power BI}
        \subsubsection{What is Power BI?}
        Power BI is a \underline{b}usiness \underline{i}ntelligence software developed by Microsoft. It is completely integrated with Office 365, which makes it very easy to import data from applications like Excel. Power BI has only been released in 2014 but was immediately seen as a competitor for tableau. Compared to Tableau, power Bi has a free desktop applications for individuals.
        
        \subsubsection{Table}
        
        \subsection{Kibana}
        \subsubsection{What is Kibana?} 
        Elastic is a search company 
        Kibana is a browser-based analytics and search dashboard for Elasticsearch, a powerful, \textbf{open source}, search engine. Although it is classified as a Monitoring tool, while the other discussed software are business intelligence software, Kibana provides very appealing and thoughtful visualizations.
        \subsubsection{Table}
        
        \subsection{Other}
        \begin{itemize}
            \item{Oracle BI}
            \item{Geckoboard}
            \item{Sisense}
            \item{Dundas Data Visualization}
            \item{SAP Business Objects}
            \item{Domo}
        \end{itemize}
        
    \section{Prototypes}
        \subsection{Prototyping software}
        The world of prototyping is a very saturated market with a broad range of tools within reach. Each tool has its strengths and weaknesses. 
    
        \begin{itemize}
            \item {Adobe XD}
            \item {Sketch}
            \item {Figma}
            \item {Invision}
            \item {Framer X}
            \item {Axure}
        \end{itemize}
   
        As the time span of this research is very limited we decided to go with Adobe XD. It is known for being easy to use. Completely free and works on Windows. 
    
    \subsection{Limitations} 
    Although prototyping software tools are rapidly improving and adding new features persistently. They are often limited to the \textbf{visual} design.
   
    One of the first \textit{Major} issues is the lack of support of \textbf{horizontal scroll bars}. Due to the large amount of columns that these tables contain, they often no longer fit into a single \gls{viewport}.
    
    Next to that there is no support for right click events, double click events, resizing components... 
    
    Although Axure supports \textit{most} of these features. It is known for having a \textbf{very steep} learning curve. Due to the limited time span and not knowing its exact capabilities, there is is no other option than to work this out with code.
    
\section{Design}
    the design here.

\section {User Scenarios}

\section{Evaluation}
    \subsection{User Feedback}
    user feedback here
    \subsection{UI Audit reports}
    ui audit reports here
    
