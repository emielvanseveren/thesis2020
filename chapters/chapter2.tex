\chapter{Research topic}

\section{Origin}
As stated in \autoref{dataminer-dashboards-information}, my internship revolves around the development within the DataMiner Dashboards application, more specifically, the usage of the generic interface. Although the generic interface provides little to no contribution to the outcome of the paper, it is the origin of the research and it provides a very interesting insight on how a large company solves certain problems. You can find a detailed explanation in \autoref{app:generic-interface}.

Currently the Dashboards application has no data analytics layer. Data can be visualized but can not be filtered or aggregated. There are utilities that provide filtering or aggregations in the front end but this is not a solution for large data sets as it is extremely slow and insufficient.
  
\section{Main research question}
The generic interface returns data in a flat table. It would be nice if this could be immediately visualized. A table is a rather simple but yet very powerful way to visualize information. But due to the large amount of data that Skyline's customers have to deal with, these tables often exist out of thousands of rows and dozens of columns, even after they are filtered. A table with these amounts of data quickly becomes inconvenient, obscure and generally not worth using. 

But how can we visualize large tabular data without sacrificing usability? This requires extensive user testing! A common technique is called prototyping. See \autoref{sec:prototyping}. We can create prototypes using prototyping software. 

After testing some of these prototyping software, I couldn't find the appropriate software for this use case. So what is the best tool to create and test large tables and the functionalities that go with it? Is there a tool that supports all the features? Where do the current tools lack?

\textbf{Note:} Although the focus is on Skyline Communications' customers, the results will be applicable to other use cases.
