\chapter{Research Skyline Communication}\label{app:skyline-communication}


\section{User Personas}
Below in \autoref{fig:user-persona-1}, \autoref{fig:user-persona-2}, \autoref{fig:user-persona-3}, you can see the user personas that represent Skyline communications' customers. More specifically those who use the DataMiner Dashboards application.

\subsection{Gitlab}
A company that uses user personas magnificently is Gitlab. Gitlab has created an entire, fictional company where the employees represent its customers. To give you an idea, Gitlab has created user personas for a compliance manager, product manager, development team lead, product designer, software developer, devOps Engineer, systems administrator, security analyst and many more. Each persona has its own job summary, motivations, frustrations, challenges...

Here's a link to Gitlab's user personas: \url{https://about.gitlab.com/handbook/marketing/product-marketing/roles-personas/#personas}


\begin{figure}[H]
        \centering
        \includegraphics[scale=0.5]{figures/user-persona-1.png}
        \caption{User persona 1}
        \label{fig:user-persona-1}
    \end{figure}
    
\begin{figure}[H]
    \centering
    \includegraphics[scale=0.5]{figures/user-persona-2.png}
    \caption{User persona 2}
    \label{fig:user-persona-2}
\end{figure}

\begin{figure}[H]
    \centering
    \includegraphics[scale=0.5]{figures/user-persona-3.png}
    \caption{User persona 3}
    \label{fig:user-persona-3}
\end{figure}

\section{Use cases}
        \begin{itemize}
            \setlength\itemsep{-0.5em}                                    
            \item {As a user I can position my data, so that the most relevant data is immediately visible.}
            \item {As a user I can control the width of a column, so that they take up less space.}
            \item {As a user I can }
        \end{itemize}
    
    \section{DataMiner Dashboards competitors}
        As a company you don't always have to reinvent the wheel. It can be very useful to see how competitors solve certain problems. But who are Skyline DataMiner Dashboards' biggest competitors? How do they visualize large tabular data?
        
        \subsection{Tableau}
        \subsubsection{What is Tableau?}
        Tableau is the leading Business Intelligence software (\gls{bis}). With Tableau users can easily retrieve, analyse, transform and report data. It enables users to effortlessly visualize data regardless of its format. The software is very easy to use and does not require any knowledge of coding. What makes Tableau so unique compared to others is their quality of data visualizations and self-service analytics. It was found in 2003 by Christian Chabot, Pat Hanrahan and Chris Stolte. In August 2019, Tableau was acquired by \href{https://www.salesforce.com/}{Salesforce} for a stunning 15 Billion dollars.
        
       \subsubsection{Table} 
        \subsection{Power BI}
        \subsubsection{What is Power BI?}
        Power BI is a business intelligence software developed by Microsoft. It is completely integrated with Office 365, which makes it very easy to import data from applications like Excel. Power BI has only been released in 2014 but was immediately seen as a competitor for tableau. Compared to Tableau, power Bi has a free desktop applications for individuals.
        
        \subsubsection{Table}
        
        \subsection{Kibana}
        \subsubsection{What is Kibana?} 
        Elastic is a search company 
        Kibana is a browser-based analytics and search dashboard for Elasticsearch, a powerful, open source, search engine. Although it is classified as a Monitoring tool, while the other discussed software are business intelligence software, Kibana provides very appealing and thoughtful visualizations.
        \subsubsection{Table}
        
        \subsection{Other}
        \begin{itemize}
            \setlength\itemsep{-0.5em}                                    
            \item{Oracle BI}
            \item{Geckoboard}
            \item{Sisense}
            \item{Dundas Data Visualization}
            \item{SAP Business Objects}
            \item{Domo}
        \end{itemize}
   